\documentclass[a4paper,12pt]{article}
\usepackage{sinhala}
\setlength{\parindent}{0in}
\begin{document}

\section*{සයුර සිංහළ යතුරුලියන සුසැටිය}
\subsection*{පණකුරු}

\begin{tabular}{|lr|lr|lr|lr|lr|lr|}
\hline
අ & a & ආ & aa & ඇ & q/A & ඈ & qq/AA & ඉ & i & ඊ & ii \\
\hline
උ & u & ඌ & uu & ඍ & U & ඎ & UU & ඏ & V & ඐ & VV \\
\hline
එ & e & ඒ & ee & ඓ & I & ඔ & o & ඕ & oo & ඖ & O \\
\hline
\end{tabular}

\subsection*{ගතකුරු}

\begin{tabular}{|lr|lr|lr|lr|lr|}
\hline
ක & k & ඛ & K/kH & ග & g & ඝ & gH & ඞ & X \\
\hline
ච & c & ඡ & C/cH & ජ & j & ඣ & J/jH & ඤ & z \\
\hline
ට & T & ඨ & TH & ඩ & D & ඪ & DH & ණ & N \\
\hline
ත & t & ථ & tH & ද & d & ධ & dH & න & n \\
\hline
ප & p & ඵ & P/pH & බ & b & භ & B/bH & ම & m \\
\hline
ය & y & ර & r & ල & l & ව & v &  ඥ & Z \\
\hline
ස & s & ශ & S & ෂ & sH/SH & හ & h & ළ & L \\
\hline
ඟ & G/gG & ඳ & dG & ඬ & DG & ඹ & M/bG & ෆ & F \\
\hline
\end{tabular}

ගතකුරකට පසු H හෝ f යෙදුමෙන් එය මහප්‍රාණයට පෙරලේ. G යෙදුමෙන් සංඤකයට පෙරලේ.

\subsection*{පිලි}

\begin{tabular}{|lr|lr|lr|lr|lr|lr|}
\hline
ක් & kw & කා & ka & කැ & kq & කෑ & kqq & කි & ki & කී & kii \\
\hline
කු & ku & කූ & kuu & කෘ & kU & කෲ & kUU & කෙ & ke & කේ & kee \\
\hline
කො & ko & කෝ & koo & කෛ & kI & කෞ & kO & කං & kx & කඃ & kQ \\
\hline
\end{tabular}

\subsection*{බැඳි අකුරු}

අල් කිරීම W ලෙස යෙදීමෙන් බැඳි අකුරු ලැබේ.

උදාහරණ: kWsH - ක්‍ෂ, nWd - න්‍ද, nWdu - න්‍දු, inWdRiy - ඉන්‍ද්‍රිය

\medskip

රකාරංසය: R, යංසය: Y

උදාහරණ: kR - ක්‍ර, kY - ක්‍ය

\bigskip
\bigskip
\bigskip

-- අනුරාධ රත්නවීර

\end{document}
